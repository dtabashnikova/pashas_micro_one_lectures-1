\documentclass{beamer}
\usepackage[russian]{babel}
\usetheme{metropolis}

\usepackage{amsthm}
\setbeamertemplate{theorems}[numbered]

\setbeamercolor{block title}{use=structure,fg=white,bg=gray!75!black}
\setbeamercolor{block body}{use=structure,fg=black,bg=gray!20!white}

\usepackage[T2A]{fontenc}
\usepackage[utf8]{inputenc}

\usepackage{hyphenat}
\usepackage{amsmath}
\usepackage{graphicx}

\setbeamercovered{transparent}

\AtBeginEnvironment{proof}{\renewcommand{\qedsymbol}{}}{}{}

\title{
Микроэкономика-I
}
\author{
Павел Андреянов, PhD
}

\begin{document}

\maketitle

\section{Квиз}

\begin{frame}{Квиз}

\begin{itemize}
  \item Чем знаменит Жерар Дебрё? \pause
  \item Почему минимум из двух вогнутых функций вогнутый? \pause
  \item Приведите пример нарушения WARP. \pause
  \item ...
  \item ...
\end{itemize}

\end{frame}

\section{Бюджетное ограничение}

\begin{frame}{План}

Первая половина лекции посвящена интерпретации Метода Множителей Лагранжа. Формулировки теорем знать необязательно, но хотелось бы, чтобы вы примерно представляли, что происходит, когда вы его применяете. 

Также будут введены термины спроса и косвенной полезности и некоторые сопутствующие определения и свойства.

Вторая половина лекции посвящена отработке техник поиска спроса и косвенной полезности во всех классических примерах.

\end{frame}


\section{Бюджетное ограничение}

\begin{frame}{Бюджетное ограничение}

Наиболее часто в нашем курсе будет встречаться линейное бюджетное ограничение:
$$ B(x,y) = p x + q y - I \leqslant 0$$

где $p, q$ - это цены товаров, а $I$ - это бюджет. 

На прошлой лекции мы уже тренировались его рисовать, опираясь на точки $(I/p, 0)$ и $(0, I/q)$, соответствующие случаю, когда все расходы тратятся на один из двух товаров.

\end{frame}

\begin{frame}{Бюджетное ограничение}

\begin{figure}[hbt]
\centering
\includegraphics[width=.8 \textwidth]{budget_2d.png}
\end{figure}

\end{frame}

\section{Метод Лагранжа}

\begin{frame}{Метод Лагранжа}

Запишем нашу оптимизационную задачу в следующем виде:
$$ U(x, y) \to \max_{(x,y) \in \mathbb{R}^2_{+}} \quad s.t.\quad  B(x,y) \leqslant 0$$

Тогда Лагранжиан принимает вид:
$$ \mathcal{L}(x, y | \lambda) = U(x,y) - \lambda B(x,y)$$

Знак перед множителем Лагранжа важен в доказательствах, но на практике не играет роли и можно ставить любой.

Традиция такова, что $\lambda I$ должен войти с плюсом.

\end{frame}

\begin{frame}{Метод Лагранжа}

Далее алгоритм предписывает найти безусловный экстремум Лагранжиана в пространстве $(x, y, \lambda)$, игнорируя ограничения.
$$ \mathcal{L}'_x = 0, \quad \mathcal{L}'_y = 0, \quad \mathcal{L}'_{\lambda} = 0.$$
Это система из трех уравнений с тремя неизвестными.

Таким образом, задача условной оптимизации сводится к безусловной. Однако не совсем понятно, почему метод Лагранжа вообще работает и что он находит.

\end{frame}

\section{Выпуклая интерпретация ММЛ}

\begin{frame}{Выпуклая интерпретация ММЛ}

Если Лагранжиан (квази) вогнутый по товарам $x,y$ то можно применить так называемый \textbf{Сильный Принцип Лагранжа}:
$$ \min_{\lambda \geqslant 0} \max_{x(\lambda),y(\lambda) \geqslant 0} \mathcal{L}(x,y | \lambda) = \textcolor{red}{\max_{x,y \geqslant 0} \min_{\lambda(x,y) \geqslant 0} \mathcal{L}(x,y | \lambda)} $$ 

Справа стоит негладкая задача, эквивалентная условной оптимизации, поскольку $\lambda(x,y)$ выбирается так, чтобы наказать бесконечно отрицательной полезностью в случае выхода за ограничение.

\end{frame}

\begin{frame}{Выпуклая интерпретация ММЛ}

Если Лагранжиан (квази) вогнутый по товарам $x,y$ то можно применить так называемый \textbf{Сильный Принцип Лагранжа}:
$$ \textcolor{red}{\min_{\lambda \geqslant 0} \max_{x(\lambda),y(\lambda) \geqslant 0} \mathcal{L}(x,y | \lambda)} =  \max_{x,y \geqslant 0} \min_{\lambda(x,y) \geqslant 0} \mathcal{L}(x,y | \lambda) $$ 

Слева стоит гладкая задача, у которой есть один экстремум типа <<седло>>, а значит его можно найти обыкновенными условиями первого порядка:
$$ \nabla_{(x,y)} \mathcal{L} = 0, \quad \nabla_{\lambda} \mathcal{L} = 0.$$

В выпуклом случае координаты решения двух задач, а также значение целевой функции совпадают. Это называется \textbf{Теоремой о Минимаксе}, или \textbf{Сильной Дуальностью}.

\end{frame}

\section{Невыпуклая интерпретация ММЛ}

\begin{frame}{Невыпуклая интерпретация ММЛ}

В общем случае, технология поиска оптимума опирается на так называемые \textbf{условия Каруш-Кун-Такера}. Основная идея такова, что градиент целевой функции и градиент активного ограничения должны быть параллельны друг другу:
$$ \nabla_{(x,y)}U - \lambda \nabla_{(x,y)} B = 0$$

Это называется необходимыми условиями первого порядка. Удивительным образом это совпадает с поиском седла Лагранжиана. Также там есть условия невязки, о которых я упомяну чуть позже.

\end{frame}

\begin{frame}{Невыпуклая интерпретация ММЛ}

В общем случае технология поиска оптимума опирается на так называемые \textbf{условия Каруш-Кун-Такера}. 

Основная идея такова, что градиент целевой функции и градиент активного ограничения должны быть параллельны друг другу:
$$ \nabla_{(x,y)}U - \lambda \nabla_{(x,y)} B = 0$$

Это называется необходимыми условиями первого порядка, или сокращенно \textbf{УПП} (в англ. \textbf{FOC}). Удивительным образом это совпадает с поиском седла Лагранжиана.

\end{frame}

\begin{frame}{Невыпуклая интерпретация ММЛ}

Далее надо сделать еще один шаг и проверить достаточные условия второго порядка, или сокращенно \textbf{УВП} (в англ. \textbf{SOC}):
$$ \nabla^2_{(x,y)}U - \lambda \nabla^2_{(x,y)} B \leqslant 0$$

на касательном к ограничении пространстве. Еще более удивительным образом это совпадает с проверкой (как бы локально) квазивогнутости Лагранжиана в точке.

\end{frame}


\section{Угловые решения}

\begin{frame}{Угловые решения}

На самом деле, поскольку мы оптимизируем в $\mathbb{R}^n_{+}$ в Лагранжиан, стоило бы добавить еще дополнительные члены, по одному на каждый товар. 
$$ \mathcal{L}(x,y | \lambda, \ldots) = U(x,y) - \lambda B(x,y) - \ldots$$

Однако в экономических приложениях, как правило, решение внутреннее. А когда оно не внутреннее, его очень легко отыскать по координатам бюджетного ограничения.

\end{frame}

\section{Значение Лагранжиана в оптимуме}

\begin{frame}{Значение Лагранжиана в оптимуме}

Вспомним условие невязки из курса мат. анализа:

$$ \lambda^{\ast} B(x^{\ast},y^{\ast}) = 0.$$

Оно означает, что одно из двух обязательно верно: либо множитель Лагранжа равен нулю, либо оптимум достигается на границе бюджетного ограничения.

Это значит, что в оптимуме значение Лагранжиана совпадает со значением целевой функцией:
$$ \mathcal{L}(x^{\ast}, y^{\ast} | \lambda^{\ast}) = U(x^{\ast}, y^{\ast}) - \lambda B(x^{\ast}, y^{\ast})$$ 

Это нам пригодится, когда мы будем изучать ее.

\end{frame}

\section{Интерпретация $\lambda$}

\begin{frame}{Интерпретация $\lambda$}

У множителя $\lambda$ в Лагранжиане есть особая интерпретация - это теневая цена нарушения ограничения:
$$\mathcal{L} = U(x,y) - \lambda B(x,y), \quad B(x,y) \leqslant 0$$ 
Если вам очень хочется выйти за ограничение, Лагранж разрешает вам это сделать, то придется дать (кому-то абстрактно) взятку размера $\lambda$. Рынок подстроится таким образом, что вы не захотите эту взятку давать. 

\end{frame}

\section{Кривые спроса}

\begin{frame}{Кривые спроса}

Нас будут интересовать координаты оптимума $x^{\ast}(p,q,I)$, $y^{\ast}(p,q,I)$ в задаче максимизации полезности при бюджетном ограничении, как функции (кривые) от цен $p,q$ и бюджета $I$. 

Они также называются \textbf{функциями (кривыми) спроса}.

\begin{definition}
Кривые вида \textbf{цена-потребление} $x^{\ast}(p,q, \ldots)$, $y^{\ast}(p,q, \ldots)$ обычно называются просто кривыми спроса. Кривые вида \textbf{доход-потребление} $x^{\ast}(I, \ldots)$, $y^{\ast}(I, \ldots)$ иногда называются кривыми Энгеля.
\end{definition}

\end{frame}

\section{Нормальные и инфериорные товары}

\begin{frame}{Нормальные товары}

Сфокусируемся на наклонах этих кривых по соответствующим параметрам. Первым мы изучим наклон кривой Энгеля, то есть кривой доход-потребление.

\begin{definition}
\textbf{Нормальными товарами} называются товары, кривые спроса которых монотонно возрастают по доходу, то есть:
$$\frac{\partial x^{\ast}}{\partial I} \geqslant 0.$$
\end{definition}
Проверка нормальности при аккуратно выведенных кривых спроса - это механическое упражнение в дифференцировании.  

\end{frame}

\begin{frame}{Инфериорные товары}

Считается, что большая часть товаров - нормальны, однако есть исключения. Например, хлеб, рис, консервы и другие товары первой необходимости иногда интерпретируются как инфериорными по отношению к красному мясу, рыбе, овощам. 

\begin{definition}
Товар, у которого нормальность нарушается хотя бы при каких-то значениях параметров, то есть 
$$\frac{\partial x^{\ast}}{\partial I} < 0,$$ 
называется \textbf{инфериорным} (при этих значениях параметров). 
\end{definition}
\end{frame}

\begin{frame}{Инфериорные товары}

Инфериорность, от англ. \textit{inferior}, означает что ваш товар $x$ является худшим по отношению к какому-то другому товару $y$. 

Когда бюджет растет, вы тратите большую часть дохода на $y$, и меньшую на $x$, да так, что в абсолютном значении потребление $x$ уменьшается. 

\begin{lemma}
Все товары не могут быть одновременно инфериорными, хотя бы один точно нормальный.
\end{lemma}

Для того, чтобы сломать нормальность $x$, обязательно должен быть хотя бы один не инфериорный товар $y$, по отношению к которому $x$ будет инфериорным.

\end{frame}

\begin{frame}{Доказательство}

Если бюджетное ограничение таково, что оптимум находится внутри, то небольшое изменение параметра $I$ не повлияет на оптимум. Если бюджетное ограничение таково, что оптимум находится на бюджетной линии, то, дифференциируя $B(x,y)= 0$ по $I$, мы получаем: 
$$ p \frac{\partial x^{\ast}}{\partial I}  + q \frac{\partial y^{\ast}}{\partial I}  = 1.$$ 

Поскольку цены неотрицательные, то инфериорность всех товаров означает, что слева стоит отрицательное число, а справа единица. Противоречие.

\end{frame}

\section{Субституты и комплементы}

\begin{frame}{Субституты}

Считается, что многие товары в той или иной степени замещаемы, некоторые больше некоторые меньше. Некоторые пары товаров особенно выделяются в этом плане, например: пепси и кола, лыжи и сноуборд... Если цена одного такого товара в паре сильно вырастет, то спрос на второй товар скорее всего вырастет. Такие товары называются субститутами.

\begin{definition}
\textbf{Субститутами} (substitutes) называются пары товаров, кривые спроса которых монотонно возрастают по ценам друг друга, то есть, $x$ субститут к $y$, если $\frac{\partial x^{\ast}}{\partial q} \geqslant 0,$ a $y$ субститут к $x$, если $\frac{\partial y^{\ast}}{\partial p} \geqslant 0.$
\end{definition}

\end{frame}

\begin{frame}{Заголовок в газетах}

\textit{Необычайная засуха в Калифорнии привела к дефициту воды и подорожанию свежих апельсинов и мандаринов на 18\%. Производители соков (не только апельсиновых, но также яблочных и других) из импортных концентратов собрались на экстренное собрание для обсуждения мер предотвращения дефицита.}

Почему они так сделали?

\end{frame}

\begin{frame}{Комплементы}

У некоторых пар товаров наблюдается прямо противоположное свойство, их обычно покупают вместе, например: кайак и весло, компьютер и монитор...  Если цена одного такого товара в паре сильно вырастет, то спрос на второй товар скорее всего упадет. Такие товары называются комплементами.

\begin{definition}
\textbf{Комплементами} (complements) называются пары товаров, кривые спроса которых монотонно убывают по ценам друг друга, то есть $x$ комплемент к $y$, если $\frac{\partial x^{\ast}}{\partial q} < 0,$ a $y$ комплемент к $x$, если $\frac{\partial y^{\ast}}{\partial p} < 0.$
\end{definition}

\end{frame}

\begin{frame}{Заголовок в газетах}

\textit{Компания Самсунг отозвала крупную партию смартфонов, в связи с браком в производстве. Чтобы удержать долю на рынке, цены на основную линейку смартфонов были уменьшены 25\%. Компания-производитель чехлов для смартфонов Самсунг неожиданно оказалась в списке единорогов.}

Что произошло?

\end{frame}

\begin{frame}{Мысли вслух}

К сожалению, субституты/комплементы не является симметричным свойством, то есть $x$ может быть субститутом к $y$, но y при этом может оказаться комплементом к $x$, хоть и в очень редких случаях. 

Это сигнализирует нам о том, что определение выбрано не совсем удачно. Мы к этому вернемся в лекции 4.

Также обратите внимание, что мы смотрели на наклоны кривых цена-потребление по не своим ценам: $x$ по $q$, $y$ по $p$. Наклон кривой спроса по собственной цене - это более сложный феномен, мы также к этому вернемся в лекции 4.

\end{frame}

\section{Косвенная полезность}

\begin{frame}{Косвенная полезность}

В каждой задаче оптимизации есть два объекта, идущие рука об руку: координаты оптимума и значение целевой функции (полезности). Мы довольно много внимания уделили координатам оптимума, то есть кривым спроса. 

А как насчет второго?

\begin{definition}
Назовем \textbf{косвенной полезностью} значение целевой функции в оптимуме в задаче максимизации полезности:
$$ V(p,q,I) = U(x^{\ast}, y^{\ast}).$$
\end{definition}
Иногда я могу также использовать символ $U^{\ast}$.

\end{frame}

\begin{frame}{Косвенная полезность}

На самом деле, не столь важно какой буквой обозначается косвенная полезность: $U^{\ast}$ или $V$. Гораздо важнее набор аргументов: $p,q, I$, подсказывающий, что координатам $x,y$ были присвоены какие-то значения в процессе оптимизации.


Внимание! В отличие от координат оптимума, косвенная полезность, конечно же зависит от всех монотонных преобразований, которые вы наложили на свою полезность.

Если вы применили преобразование, например, $\log x$, чтобы быстрее решить задачу, и получили косвенную полезность, то вам придется ее откатить, то есть применить к ней обратное преобразование $e^x$, чтобы ответ был формально верным.

\end{frame}

\section{Непрерывность спроса}

\begin{frame}{Непрерывность спроса}

В большей часть примеров, которые мы будем рассматривать, спросы будут выражаться через элементарные функции, такие как $x^2, \log x, 1/x$... Все эти функции непрерывны. Совпадение?

Ответить на этот вопрос нам поможет \textbf{Теорема Максимума}: в \textbf{строго выпуклой} задаче оптимизации, решение, если оно существует, то единственно. Более того, если \textbf{задача непрерывна} по параметрам, то как координаты оптимума, так и значение целевой функции непрерывны по параметрам.

Я буду иногда пользоваться следующими неформальными определениями: в строго выпуклой задаче, функция $U$ (квази) вогнутa, функция $B$ (квази) выпукла, причем одна из двух строго. В непрерывной задаче обе функции $U$ и $B$ непрерывны.

\end{frame}

\section{Перерыв}

\section{Кривая Энгеля}

\begin{frame}{Кривая Энгеля}

Если взять две кривые доход-потребление: $x^{\ast}(I), y^{\ast}(I)$, то получится параметрически заданная кривая в пространстве товаров $(x,y)$. 

Вот эта кривая и называется кривой Энгеля.

\end{frame}

\section{Кобб-Дуглас}

\begin{frame}{Кобб-Дуглас}

\begin{definition}
Полезностью \textbf{Коббп-Дугласа} называется:
$$U(x, y) = x^{\alpha} y^{1-\alpha}, \quad \alpha \in (0,1)$$  
\end{definition}

Вспомним, что монотонные преобразования полезности не меняют поведение потребителя. Тогда можно применить логарифм и получить:
$$ U(x, y) = \alpha \log x + (1-\alpha) \log y, \quad \alpha \in (0,1).$$ 
Заметим, что эта функция вогнута!!! 
\end{frame}

\begin{frame}{Кобб-Дуглас}

Выпишем Лагранжиан:
$$ \mathcal{L} = \alpha \log x + (1-\alpha) \log y - \lambda (px + qy -I).$$ 

Заметим, что я выставляю знак минус так, чтобы у множителя Лагранжа была интерпретация теневой цены выхода за бюджетное ограничение. Это нам пригодится в следующей лекции, а сейчас просто постарайтесь запомнить.
\end{frame}

\begin{frame}{Кобб-Дуглас}

Бездумно выпишем три уравнения:

$\mathcal{L}'_x = \alpha/ x - \lambda p = 0$

$\mathcal{L}'_y = (1-\alpha)/y - \lambda q = 0$

$\mathcal{L}'_{\lambda} = I - p x - qy = 0$

Легко видеть, что они эквивалентны

$\alpha - \lambda p x= 0$

$(1-\alpha) - \lambda q y= 0$

$px + qy - I = 0$

\end{frame}

\begin{frame}{Кобб-Дуглас}

Обозначим доли бюджета потраченные на $x$ и $y$ как $s_x= px$ и $s_y = qy$ соответственно, и умножим последнее уравнение на $\lambda$. 

Тогда уравнения становятся еще проще:

$\alpha = \lambda s_x$

$(1-\alpha) = \lambda s_y$

$\lambda s_x + \lambda s_y = \lambda I$

Эту систему можно уже решить в уме. 

Получается, что теневая цена равна $\lambda = 1/I$, а доли бюджета, потраченные на каждый товар, постоянны и равны $\alpha$ и $1-\alpha$.

Это интуитивно?

\end{frame}

\begin{frame}{Кобб Дуглас}

Пусть полезность имеет следующий вид:
$$U(x,y,z) = \alpha \log x + \beta \log y + \gamma \log z$$ 
а цены равны $p, q, r$ соответственно.

Спрос на каждый товар в Коббе-Дугласе описывается следующими уравнениями:
\begin{gather*}
x^{\ast} = \frac{\alpha}{\alpha + \beta + \gamma} \frac{I}{p}, \quad
y^{\ast} = \frac{\beta}{\alpha + \beta + \gamma} \frac{I}{q}, \quad
z^{\ast} = \frac{\gamma}{\alpha + \beta + \gamma} \frac{I}{r}
\end{gather*}

Такое лучше запомнить наизусть. Также постарайтесь ответить, являются ли такие товары нормальными, комплементами или субститутами.

\end{frame}

\begin{frame}{Кобб Дуглас}

Нампомним, что косвенная полезность чувствительна к монотонным преобразованиям, поэтому тут важно какая именно спецификация была изначально дана в задаче. 

Для простоты давайте считать, что это спецификация в логарифмах.

Сосчитаем логарифм спроса на первый товар:
$$\log x^{\ast} = \log \alpha - \log (\alpha + \beta + \gamma) + \log I - \log p$$
Аналогично считается логарифм спроса на другие товары. Теперь надо просто подставить их в полезность.

\end{frame}

\begin{frame}{Кобб Дуглас}

Косвенная полезность в Коббе-Дугласе (с точностью до преобразования) имеет вид
$$V(p,q,r,I) = (\alpha + \beta + \gamma) \log I - \alpha \log p - \beta \log q - \gamma \log r + C $$
Константы $C$ можно, как правило, не выписывать, так как они исчезнут при первой же попытке продифференцировать.

Эта формула нам будет очень полезна в будущем...
\end{frame}

\section{Леонтьев}

\begin{frame}{Леонтьев}

\begin{definition}
Полезностью \textbf{Леонтьева} называется:
$$U(x, y) = \min(x/a, y/b)$$  
\end{definition}

Интерпретация полезности такая, что для извлечения одной единицы полезности необходимо ровно a и b единиц потребительских товаров. Иногда такая полезность называется \textbf{совершенными комплементами}.

\end{frame}

\begin{frame}{Леонтьев}

\begin{figure}[hbt]
\centering
\includegraphics[width=.8 \textwidth]{leontiev}
\end{figure}

\end{frame}

\begin{frame}{Леонтьев}

Поскольку задача негладкая, то геометрический метод проще и быстрее. Решение лежит в пересечении кривой Энгеля и бюджетной линии. 

Соответственно, достаточно решить систему уравнений:
$$ px + qy = I, \quad b x = a y$$
Кривая Энгеля здесь – это множество точек, от которых отложены уголки.

\end{frame}

\begin{frame}{Леонтьев}

Пусть полезность имеет следующий вид:
$$U(x,y,z) = \min(x/a, y/b, z/c)$$ 
а цены равны $p, q, r$ соответственно. 

Спрос на каждый товар в Леонтьеве описывается следующими уравнениями:
\begin{gather*}
x^{\ast} = \frac{ap}{ap + bq + cr} \frac{I}{p}, \\
y^{\ast} = \frac{bq}{ap + bq + cr} \frac{I}{q}, \\
z^{\ast} = \frac{cr}{ap + bq + cr} \frac{I}{r}.
\end{gather*}

Все товары в функции Леонтьева являются нормальными, а также попарно являются (строго) комплементами.

\end{frame}

\begin{frame}{Леонтьев}

Заметим, что в оптимуме полезности в обоих позициях аргумента одинаковые. То есть косвенная полезность равна одновременно левому и правому аргументу.


Косвенная полезность в Леонтьеве (с точностью до преобразования) имеет вид
$$V(p,q,I) = \frac{I}{ap + bq + cr}$$

Это тоже очень полезная формула.

\end{frame}

\section{Квазилинейная}

\begin{frame}{Квазилинейная}

Пожалуй, третья самая важная полезность имеет следующий вид:

\begin{definition}
\textbf{Квазилинейной полезностью} называется:
$$U(x, y) = f(x) + k y,$$ 
где $f$ - вогнутая функция.
\end{definition}

Интерпретация последней координаты - это деньги на счету. То есть вам не обязательно тратить весь бюджет как раньше и остаток средств на счету конвертируется в утили по курсу 1:$k$.

\end{frame}

\begin{frame}{Квазилинейная}

Выпишем Лагранжиан:
$$\mathcal{L} = f(x) + k y - \lambda (px + y - I).$$ 
Легко, правда?

Обратите внимание, что цена денег равна единице.

\end{frame}

\begin{frame}{Квазилинейная}

Сейчас мы попробуем найти внутреннее решение.

$\mathcal{L}'_x = f'_x - \lambda p = 0$

$\mathcal{L}'_y = k - \lambda = 0$

$\mathcal{L}'_{\lambda} = I - p x - y= 0$

Легко видеть, что они эквивалентны

$k = \lambda$

$x = (f')^{-1}(\lambda p)$

$px + y = I$

\end{frame}

\begin{frame}{Квазилинейная}

Сейчас мы попробуем найти внутреннее решение.

$\mathcal{L}'_x = f'_x - \lambda p = 0$

$\mathcal{L}'_y = k - \lambda = 0$

$\mathcal{L}'_{\lambda} = I - p x - y= 0$

Легко видеть, что они эквивалентны

$k = \lambda$

$x = (f')^{-1}(\lambda p)$

$px + y = I$

\end{frame}

\begin{frame}{Квазилинейная}

Однако эта система не всегда имеет решение в $\mathbb{R}^2_{+}$. Легко видеть, что спрос на товар $x$ никак не зависит от бюджета, а стало быть, при достаточно маленьком бюджете спрос на товар $y$ упрется в ноль.

Мы оказались в ситуации, о которой я предупреждал. Условия первого порядка указали на точку, которая может оказаться вне допустимой области. Если это так, это значит что решение не внутреннее, а краевое. В таком случае, мы заменяем условие первого порядка  $x = (f')^{-1}(\lambda p)$ на краевое условие $y=0$ или эквивалентно $x = I/p$.

\end{frame}


\begin{frame}{Квазилинейная}

В этой задаче есть два взаимоисключающих режима: внутреннее решение и краевое решение. Но вместо перебора случаев, можно записать ответ в компактной форме, если проявить немного смекалки.

Спрос на каждый товар в квазилинейной полезности описывается следующими уравнениями:
\begin{gather*}
x^{\ast} = \min (I/p, (f')^{-1}(k p)), \\
y^{\ast} = \max (0, I-px^{\ast}).
\end{gather*}
Все товары в квазилинейной полезности являются нормальными, a деньги (переменная $y$) являются универсальным комплементом.
\end{frame}

\begin{frame}{Квазилинейная}

Поскольку в задаче два режима, скорее всего ответ будет иметь форму максимума или минимума из двух выражений. Если бы ограничения не было, решение было бы всегда внутреннее, а полезность равна 
$$f((f')^{-1}(k p)) + I - p (f')^{-1}(k p).$$

Когда ограничение активно, оно мешает нам достигнуть этой полезности и мы получаем вместо нее
$$ f(I/p) + 0.$$

\end{frame}

\section{Линейная}

\begin{frame}{Линейная}

Простая с виду, но очень неудобная на практике:

\begin{definition}
\textbf{Линейной полезностью} называется:
$$U(x, y) = x/a +y/b,$$ 
\end{definition}

интерпретируется как способность извлекать одну и туже полезность из разных источников.  Конкретно вы можете получить одну единицу полезности либо из $a$ единиц товара $x$, либо из $b$ единиц товара $y$. 

Это значит, что $x, y$ обладают высокой взаимозаменяемостью либо вообще представляют собой один и тот же товар в пачках/таре разного размера. Такая полезность еще часто называется \textbf{совершенными субститутами}.

\end{frame}

\begin{frame}{Линейная}

Решение в этой задаче не похоже на предыдущие, оно вообще всегда краевое. 

Почему так? Посмотрим внимательно на бюджетное ограничение:

$$B(x,y) = px + qy - I \leqslant 0$$ 

оно показывает, что вы можете менять товары $x, y$ по курсу $\frac{1}{p}$ к $\frac{1}{q}$. А в полезности вы можете менять товары по курсу $a$:$b$. За исключением редкого случая, когда эти курсы совпадают: 
$$ap = bq,$$ 
вам выгодно менять один товар на другой до упора.
\end{frame}

\begin{frame}{Линейная}

Осталось понять, каким будет краевое решение...

Интуитивно понятно, что вы будете тратить все на $x$, когда его вес в полезности относительно большой, а его цена относительно маленькая. То есть, когда $ap$ относительно маленький. 

Относительно чего? Конечно же, относительно $bq$.

\end{frame}

\begin{frame}{Линейная}

Осталось понять, каким будет краевое решение...

Интуитивно понятно, что вы будете тратить все на $x$, когда его вес в полезности относительно большой, а его цена относительно маленькая. То есть, когда $ap$ относительно маленький. 

Относительно чего? Конечно же, относительно $bq$.

Спрос на каждый товар описывается так: 

если $ap < bq$, то $x^{\ast} = I/p, y^{\ast} = 0$

если $ap > bq$, то $x^{\ast} = 0, y^{\ast} = I/q$

Все товары в линейной полезности нормальные и являются попарно субститутами.

\end{frame}

\begin{frame}{Линейная}

Мы знаем, что решение либо в одном углу, либо в другом. Соответственно, ответ это наибольшая из двух полезностей этих кандидатов, то есть
$$V(p,q,I) = I \cdot \max(\frac{1}{ap}, \frac{1}{bq}).$$
Пользуясь тем, что максимум коммутирует с монотонно возрастающими преобразованиями
$$ \varphi'(x) >0 \quad \Rightarrow \quad \max(\varphi(x), \varphi(x)) = \varphi(\max(x, y)$$
и с монотонно убывающими преобразованиями в некотором смысле тоже
$$ \psi'(x) < 0 \quad \Rightarrow \quad \max(\psi(x), \psi(x)) = \psi(\min(x, y)$$
можно вывести следующее красивое свойство...

\end{frame}

\begin{frame}{Линейная}

Косвенная полезность в линейной полезности (с точностью до преобразования) имеет вид
$$V(p,q,I) = I / \min(ap, bq),$$
Это тоже лучше запомнить наизусть.
\end{frame}

\section{Конец}

\end{document}